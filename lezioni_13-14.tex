\documentclass[12pt]{article}

\usepackage{amsmath}
\usepackage{amssymb}
\usepackage{mathtools}
\usepackage{graphicx}

\usepackage[utf8]{inputenc}

\newcommand {\R}{\mathbb{R}}
\newcommand {\N}{\mathbb{N}}

\begin{document}


\section{Forme Quadratiche}

\subsection{Definizione}

Sia $A\in \R ^ {n\times n}$ $A=A^T$ considero $q_A:\; \R^n \rightarrow \R $ $q_A(h) = \langle Ah, h\rangle$\newline
$\forall h = (h_1,\dots, h_n)\in \R$ \qquad $A\in \R^{n\times n},\;h\in \R^{n\times 1},\; Ah\in \R^{n\times 1}$\newline
$q_A$ è la forma quadratica associata alla matrice quadrata e simmetrica A

\textbf{quadrata:} matrice che ha lo stesso numero di righe e colonne

\textbf{simmetrica:} matrice che è uguale alla sua trasposta


$$
    A =
    \begin{bmatrix}
        a & b \\
        b & c
    \end{bmatrix}
    = A^T
$$
$$
    q_A = \;\langle
    \begin{bmatrix}
        a & b \\
        b & c
    \end{bmatrix}
    \begin{bmatrix}
        h_1 \\
        h_2
    \end{bmatrix},
    \begin{bmatrix}
        h_1 \\
        h_2
    \end{bmatrix}
    \rangle\; = \;\langle
    \begin{bmatrix}
        ah_1 + bh_2 \\
        bh_1 + ch_2
    \end{bmatrix},
    \begin{bmatrix}
        h_1 \\
        h_2
    \end{bmatrix}
    \rangle\; = ah_1^2+2b\,h_1\,h_2+ch_2^2
$$
\\Caso con n generico:

$$
    q_A = \sum_{j,k = 1}^{n} a_{jk} h_k h_j = \sum_{j = 1}^{n} a_{jj} h_j^2 + \sum_{1\leq j < k\leq n} a_{jk}h_jh_k
$$

\paragraph*{Osservazione informale:}
Abbiamo trovato un polinomio di grado 2,\\
quindi possiamo dire che le forme quadratiche sono delle funzioni
associate a delle matrici che rappresentano polinomi


\subsection{Segno di una forma quadratica}

\textbf{Definizione:} $A^T = A \in \R^{n\times n}$

\begin{enumerate}
    \item Si dice che A è definita positiva se vale $\langle Ah, h\rangle > 0\; \forall h \neq 0 \in \R ^n$
    \item Si dice che A è definita negativa se vale $\langle Ah, h\rangle < 0\; \forall h \neq 0 \in \R ^n$
    \item Si dice che A è indefinita se $\exists h^+, h^- \in \R ^n$ t.c. \\$\langle Ah^-,h^-\rangle  \lneqq 0 \lneqq \langle Ah^+,h^+ \rangle$
\end{enumerate}

\paragraph*{Osservazione informale:}
La matrice A è positiva se per ogni vettore h è positiva, stessa cosa vale per il negativo.
Invece si dice indefinita se per alcuni vettori h è negativa e per altri è positiva,
quindi non possiamo assegnarli un segno preciso.

\paragraph*{Osservazione informale:}
I segni di disuguaglianza devono essere stretti ($<, >$),
altrimenti si dice che A è semidefinita positiva.\\


Forme quadratiche non singolari:
\begin{enumerate}
    \item $A > 0 \Leftrightarrow
              \begin{cases}
                  a > 0                                             \\
                  ac-b^2 > 0 & \quad \textit{determinante positivo}
              \end{cases} $
    \item $A < 0 \Leftrightarrow
              \begin{cases}
                  a < 0                                             \\
                  ac-b^2 > 0 & \quad \textit{determinante positivo}
              \end{cases} $
    \item A è indefinita $\Leftrightarrow ac-b^2<0 \quad \textit{determinante negativo}$
          \\
          \\Forme quadratiche singolari:
    \item se $ac-b^2 = 0$, quindi \textit{determinante nullo}, si tratta di una matrice singolare, quindi A è semidefinita
\end{enumerate}

\subsection{Proposizione}

Se $A = A^T \in \R^{n\times n}$ è definita positiva,
allora $\exists m >0$ t.c. $$\langle Ah, h\rangle \geq m \lvert h \rvert ^2 \quad \forall h \in \R$$
Allo stesso modo se A è definita negativa, allora $\exists m >0$ t.c. $$\langle Ah, h\rangle \leq m \left\lvert h \right\rvert ^2 \quad \forall h \in \R$$

\paragraph*{Dimostrazione: (n=2)}
Scriviamo $h=(r\cos\theta , r\sin\theta)$ con $r\geq 0, r = \lvert h\rvert$ e $\theta \in [0, 2\pi]$\\
Allora vale $\langle Ah, h\rangle = a_{11}\,r^2\cos^2 \theta\;+\; 2a_{12}\,r^2\cos\theta\sin\theta\;+\;a_{22}\,r^2\sin^2\theta = r^2[a_{11}\cos^2 \theta\;+\; 2a_{12}\cos\theta\sin\theta\;+\;a_{22}\sin^2\theta]$\\
Poniamo $g(\theta) = [\dots]$ per $\theta \in [0, 2\pi]$\\
Per ipotesi $g(\theta) > 0 \quad \forall \theta \in [0, 2\pi]$ (infatti $r^2g(\theta)>0 \quad \forall r>0$ e $\theta \in [0, 2\pi]$)\\
Essendo f continua su $[0, 2\pi]$ per il teorema di Weistrass $\exists \overline{\theta} \in [0, 2\pi]$ tale che $g(\overline{\theta}) = min\;g$.\\
Tale minimo è positivo e lo chiamiamo m. Dunque $\langle Ah, h\rangle = r^2g(\theta) \geq r^2m = m \lvert h \rvert^2 \quad \forall h$\\


\section{Formula di Taylor di ordine 2}

$A\subseteq \R^n$ aperto, $f:\;A\rightarrow \R$, f è di classe $C^2$\\
Allora vale $\forall \overline{x} \in A$ vale lo sviluppo
$$f(\overline{x} + h) = f(\overline{x}) + \langle \bigtriangledown f(\overline{x}), h \rangle + \frac{1}{2}\langle Hf(\overline{x})h, h \rangle + o(\rvert h \rvert ^2)\;\;\;\text{per } h\rightarrow 0$$

\paragraph*{Dimostrazione:}
Dimostriamo la seguente formula con resto "non uniforme"
$$\forall v \in \R ^n,\; \rvert v\rvert = 1, \forall x \in A$$
vale la formula
\begin{equation}\label{eq:taylor2}
    f(\overline{x} + tv) = f(\overline{x}) + \langle \bigtriangledown f(\overline{x}), tv \rangle + \frac{1}{2}\langle Hf(\overline{x})tv, tv \rangle + o(t ^2)\;\;\;\text{per } t\rightarrow 0 \in \R
\end{equation}
\\
Consideriamo la funzione $g:\;]-\varepsilon, \varepsilon[ \;\rightarrow \R$, $g(t) = f(\overline{x} + tv)$
            definita per $\varepsilon$  sufficientemente piccolo. \\
            Poichè f è di classe $C^2$, si vede che
        $\exists g'(t) = \langle \bigtriangledown f(\overline{x}+tv), v \rangle\;\;\forall t \in \;]-\varepsilon, \varepsilon[$
inoltre esiste ed è  continua $g''(t) = \langle Hf(\overline{x}+tv)v, v \rangle$
\\\\
Scriviamo la Taylor in t per g con punto iniziale t = 0. Otteniamo:
$$g(t) = g(0) + g'(0)t + g''(0)\frac{t^2}{2} + o(t^2)$$
Trascrivendo in termini di f si trova esattamente la formula~\ref{eq:taylor2} da dimostrare.

\section{Teorema di classificazione dei punti critici}

Se $f:\; A \rightarrow \R$ è $C^2$ sull'aperto $A\subseteq \R ^n$, vale quanto segue, per $\overline{x} \in A$

\begin{enumerate}
    \item $
              \begin{cases}
                  \bigtriangledown f(\overline{x}) = 0                                                      \\
                  Hf(\overline{x}) > 0 & \quad\Longrightarrow \overline{x} \text{ è punto di minimo locale}
              \end{cases}
          $
    \item $
              \begin{cases}
                  \bigtriangledown f(\overline{x}) = 0                                                       \\
                  Hf(\overline{x}) < 0 & \quad\Longrightarrow \overline{x} \text{ è punto di massimo locale}
              \end{cases}
          $
    \item $
              \begin{cases}
                  \bigtriangledown f(\overline{x}) = 0                                                             \\
                  Hf(\overline{x}) \text{ indefinita} & \quad\Longrightarrow \overline{x} \text{ è punto di sella}
              \end{cases}
          $
\end{enumerate}

\paragraph*{Nota:}
$\overline{x}$ punto critico di f si dice di sella se $\forall r > 0\;\; \exists x_+\,,\; x_- \in B(\overline{x}, r)$
tale che $f(x_-) < f(\overline{x}) < f(x_+)$



\end{document}
