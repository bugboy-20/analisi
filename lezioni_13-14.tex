\documentclass[12pt]{article}

\usepackage{amsmath}
\usepackage{amssymb}
\usepackage{mathtools}
\newcommand {\R}{\mathbb{R}}
\newcommand {\N}{\mathbb{N}}
\usepackage{graphicx}

\usepackage[utf8]{inputenc}


\begin{document}


\section{Forme Quadratiche}

\subsection{Definizione}

Sia $A\in \R ^ {n\times n}$ $A=A^T$ considero $q_A:\; \R^n \rightarrow \R $ $q_A(h) = <Ah, h>$\newline
$\forall h = (h_1,\dots, h_n)\in \R$ \qquad $A\in \R^{n\times n},\;h\in \R^{n\times 1},\; Ah\in \R^{n\times 1}$\newline
$q_A$ è la forma quadratica associata alla matrice quadrata e simmetrica A

\textbf{quadrata:} matrice che ha lo stesso numero di righe e colonne

\textbf{simmetrica:} matrice che è uguale alla sua trasposta


$$
    A =
    \begin{bmatrix}
        a & b \\
        b & c
    \end{bmatrix}
    = A^T
$$
$$
    q_A = <
    \begin{bmatrix}
        a & b \\
        b & c
    \end{bmatrix}
    \begin{bmatrix}
        h_1 \\
        h_2
    \end{bmatrix},
    \begin{bmatrix}
        h_1 \\
        h_2
    \end{bmatrix}
    > = <
    \begin{bmatrix}
        ah_1 + bh_2 \\
        bh_1 + ch_2
    \end{bmatrix},
    \begin{bmatrix}
        h_1 \\
        h_2
    \end{bmatrix}
    > = ah_1^2+2b\,h_1\,h_2+ch_2^2
$$

\paragraph{Osservazione informale:}
Abbiamo trovato un polinomio di grado 2,\\
quindi possiamo dire che le forme quadratiche sono delle funzioni
associate a delle matrici che rappresentano polinomi


\subsection{Segno di una forma quadratica}



\end{document}