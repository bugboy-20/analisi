\documentclass[12pt]{article}

\usepackage{amsmath}
\usepackage{amssymb}
\usepackage{mathtools}
\usepackage{graphicx}

\usepackage[utf8]{inputenc}

\newcommand {\R}{\mathbb{R}}
\newcommand {\N}{\mathbb{N}}

\begin{document}


\section{Forme Quadratiche}

\subsection{Definizione}

Sia $A\in \R ^ {n\times n}$ $A=A^T$ considero $q_A:\; \R^n \rightarrow \R $ $q_A(h) = \langle Ah, h\rangle$\newline
$\forall h = (h_1,\dots, h_n)\in \R$ \qquad $A\in \R^{n\times n},\;h\in \R^{n\times 1},\; Ah\in \R^{n\times 1}$\newline
$q_A$ è la forma quadratica associata alla matrice quadrata e simmetrica A

\textbf{quadrata:} matrice che ha lo stesso numero di righe e colonne

\textbf{simmetrica:} matrice che è uguale alla sua trasposta


$$
    A =
    \begin{bmatrix}
        a & b \\
        b & c
    \end{bmatrix}
    = A^T
$$
$$
    q_A = \;\langle
    \begin{bmatrix}
        a & b \\
        b & c
    \end{bmatrix}
    \begin{bmatrix}
        h_1 \\
        h_2
    \end{bmatrix},
    \begin{bmatrix}
        h_1 \\
        h_2
    \end{bmatrix}
    \rangle\; = \;\langle
    \begin{bmatrix}
        ah_1 + bh_2 \\
        bh_1 + ch_2
    \end{bmatrix},
    \begin{bmatrix}
        h_1 \\
        h_2
    \end{bmatrix}
    \rangle\; = ah_1^2+2b\,h_1\,h_2+ch_2^2
$$
\\Caso con n generico:

$$
    q_A = \sum_{j,k = 1}^{n} a_{jk} h_k h_j = \sum_{j = 1}^{n} a_{jj} h_j^2 + \sum_{1\leq j < k\leq n} a_{jk}h_jh_k
$$

\paragraph*{Osservazione informale:}
Abbiamo trovato un polinomio di grado 2,\\
quindi possiamo dire che le forme quadratiche sono delle funzioni
associate a delle matrici che rappresentano polinomi


\subsection{Segno di una forma quadratica}

\textbf{Def:} $A^T = A \in \R^{n\times n}$

\begin{enumerate}
    \item Si dice che A è definita positiva se vale $\langle Ah, h\rangle > 0\; \forall h \neq 0 \in \R ^n$
    \item Si dice che A è definita negativa se vale $\langle Ah, h\rangle < 0\; \forall h \neq 0 \in \R ^n$
    \item Si dice che A è indefinita se $\exists h^+, h^- \in \R ^n$ t.c. \\$\langle Ah^-,h^-\rangle  \lneqq 0 \lneqq \langle Ah^+,h^+ \rangle$
\end{enumerate}

\paragraph*{Osservazione informale:}
La matrice A è positiva se per ogni vettore h è positiva, stessa cosa vale per il negativo.
Invece si dice indefinita se per alcuni vettori h è negativa e per altri è positiva,
quindi non possiamo assegnarli un segno preciso.

\paragraph*{Osservazione informale:}
I segni di disuguaglianza devono essere stretti ($<, >$),
altrimenti si dice che A è semidefinita positiva.\\


Forme quadratiche non singolari:
\begin{enumerate}
    \item $A > 0 \Leftrightarrow
              \begin{cases}
                  a > 0                                             \\
                  ac-b^2 > 0 & \quad \textit{determinante positivo}
              \end{cases} $
    \item $A < 0 \Leftrightarrow
              \begin{cases}
                  a < 0                                             \\
                  ac-b^2 > 0 & \quad \textit{determinante positivo}
              \end{cases} $
    \item A è indefinita $\Leftrightarrow ac-b^2<0 \quad \textit{determinante negativo}$
          \\
          \\Forme quadratiche singolari:
    \item se $ac-b^2 = 0$, quindi \textit{determinante nullo}, si tratta di una matrice singolare, quindi A è semidefinita
\end{enumerate}




\end{document}