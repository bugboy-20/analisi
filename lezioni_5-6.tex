\documentclass[a4paper]{article}

\newcommand {\R}{\mathbb{R}}
\newcommand {\N}{\mathbb{N}}
\begin{document}

\section{Spazio euclideo}
$$
\R ^ n := \{ x = (x_1, x_2, \dots , x_n | x_1, x_2, x_n \in \R \}
$$	
In $ \R ^ n $ vale 

\begin{description}
	\item [Somma tra vettori] $ x = (x_1,\dots,x_2) , y = (y_1, ... , y_n) $
		$$
		x+y=(x_1+y_1 + \dots + x_n + y_n )
		$$
	\item [Prodotto  con scalare] dato $ x=(x_1, \dots , x_n) , \lambda \in \R $, poniamo
		$$
		\lambda x:=( \lambda x_1, \dots , \lambda x_n )
		$$
\end{description}

\paragraph{Definizione Prodotto scalare euclideo} Dati $x , y \in \R ^ n $, poniamo:
$$
\langle x,y \rangle := \sum_{k=1}^n x_k y_k
$$

\subsection{ Proprietà: }

\begin{enumerate}
	\item %[Simmetrie] 
		$\langle x,y \rangle = \langle y,x \rangle \quad \forall x,y \in \R ^n $ 

	\item %[Bilinearità] 
		$ \langle \lambda x + \mu y , z \rangle = \lambda \langle x ,z \rangle + \mu \langle y,z \rangle $ e
		$ \langle z, \lambda x + \mu y \rangle = \lambda \langle z,x \rangle + \mu \langle z,y \rangle \quad \forall x,y,z \in \R ^ n \et \lambda , \mu \in \R $
		\item $ \langle x,x \rangle \ge 0 \quad \forall x \in \R ^ n $
		\item $ \langle x,x \rangle = 0 \iff x= \underline{0} = (0,0,\dots, 0). $
\end{enumerate}

\subsection { Definizione Vettori ortogonale} $x,y \in \R ^n $ si dicono ortogonali se $ \langle x,y \rangle = 0$

\subsection { Definizione Norma euclidea }
Dato $ x \in \R ^n $, poniamo $ \|x \| := \sqrt{\langle x,x \rangle } \in [0, + \infty [$

Si dice norma di x (viene usata la notazione $|x|$)

\paragraph { Interpretazione della norma con lunghezza (con il Teorema di Pitagora) }

\subsubsection{Proprietà della norma}

\begin{enumerate}
	\item $ | \lambda x | = | \lambda | \cdot | x | \quad \forall \lambda \in \R , x \in \R ^ n $
	\item $ | x | \ge 0 \quad \forall x \in \R ^ n $ in oltre $ | x| =0 \iff x=0 $
	\item $ | x+y | \le |x| + |y| \quad forall x,y \in \R ^n$ (disuguanza triangolare, con relativa interpretazione)
\end{enumerate}

\subsection {Normalizzato di un vettore}
\paragraph{ Definizione: } dato $x \neq 0, x \in \R ^n, $ il normalizzato di x è il vettore $ x \over | x | $, l'unico multiplo positivo di $x$ che ha norma 1

\subsection{ Scrittura del prodotto scalare in coordinate polati in $ \R ^ n $}
Dati $ x \in \R ^ 2 \setminus \{ 0 \} $ , scriviamo 
$$
x= | x | { x \over |x| } = r ( \cos \theta , \sin \theta )
$$
dove $r=|x|$ e $\theta \in \R$ è opportuno. Presi $x = ( r \cos \theta, r \sin \theta)$ e $y=( \rho \cos \phi , \rho \sin \phi )$ , risulta
$$
\langle x,y \rangle = r \rho \cos ( \phi - \theta ) = |x| \cdot |y| \cos ( \phi - \theta ) 
$$
la conseguenza è la disuguaglianza di Clauchy-Schwarz

\subsection{ La disuguaglianza di Clauchy-Schwarz}

$\forall x,y \in \R ^ n $ vale 
$$
| \langle x,y \rangle | \le |x| \cdot |y|
$$
Inoltre vale l'uguaglianza sse $x$ e $y$ sono indipendenti

\subsection { Formula del "quadrato di un binomio " }
Dati $ x,y \in \R ^ n $ vale $$ | x+y | ^2 = |x| ^ 2 + 2 \langle x,y \rangle + |y|^2 $$
La dimostazione avviene con le proprietà del prodotto scalare.
Dalla formula sopra segue che, se $ x \perp y $ in $ \R ^ n $, allora vale
$$
| x +y | ^ 2 = | x | ^ 2 + | y | ^ 2 
$$ 
\textbf { Teorema dio Pitagora }

\subsection { Disuguaglianza triangolare }
Ancora della formula del "quadrato di un binomio" si può ottenere la dimostazione della disuguaglianza triangolare
$$
|x+y| \le |x| + |y| \quad \forall x,y \in \R ^ n 
$$
Infatti
$$
| x+y |^2 = |x|^2 + |y|^2 +2 \langle x, y \rangle \\
\le \text{ (per Clauchy-Schwarz)} \le |x|^2 + |y|^2 +2 | x | \cdot | y| \\
= \left ( | x|+|y| \right ) ^2 \quad \forall x,y \in \R ^ n
$$
\subsection{Definizione distanza}
$ \forall x,y \in \R ^ $ la distanza tra $x$ e $y$ è 
$$
| x-y |
$$

\subsection{ Intorni sferici o dischi o palle }
Dato $ x \in \R ^ n $ (centro) e $ r > 0 $ (raggio), poniamo
$$
B(x,r)= \left \{ y \in \R ^ n \mid |y-x| < r \right \} \mbox{(palla con centro x e raggio r)}
$$

\subsection{ Definizione insieme limitato } 
Sia $A \subseteq \R ^n $ , si dice limitato se $ \exists R > 0 $ t.c $ A \subseteq B(0,R) $ 

\subsection { Insieme aperto } 
Sia $ A \subseteq \R ^ n $ si dice aperto se
$$
\forall x \in A \exists r > 0 \mbox{ t.c } B(x,r) \subseteq A
$$
\paragraph{Esempi: } Gli intervalli $]a,b[$, i rettangoli $ A = I \time J \subseteq \R ^ 2 $ con $I,J$ aperti in \R . 

\section{ Sucessioni in \R ^ n }
Sia $ (x_k)_{k \in \N} $ una sucessione in $\R ^ n \quad \forall k \in \N$
\subsection {Definizione} 
$ (x_k)_{k \in \N} $ sucessione in $ \R ^ n$ ; $ x \in \R ^ n $
Si dice $ x_k \to x $ per $ k \to + \infty $ se vale 
$$
\lim_{k \to + \infty} x^j_k = x^j \quad \forall j \in \{ 1,2, \dots , n \}
$$
\paragraph {  Equivalentemente} se vale  $ lim_{k \to + \infty} | x_k -x | =0

\section{Funzioni di più variabili} 
$ A \subseteq \R ^ n, B \subseteq \R ^ q$. Data $ f: A \to B$ , il grafico di f è
$$
Graf(G)= \{ (x,f(x)) \mid x \in A \} \subseteq A \times B 
$$

\subsection { Definizione funzione continua} $f: A \to B$ (con $ A \subseteq \R^n , B \subseteq \R ^ q $)

f si dice continua se $ \overline x $ se vale quanto segue:
$$
\forall  (x_k)_{k \in \N} \mbox{, $(x_k)$ sucessione in A, } x_k \xrightarrow[k \to + \infty]{ } \overline x \\ 
\implies f( x_k ) \to f( \overline x ) \quad k \to + \infty
$$

Si dimostra che la definizione di continuà "per sucessioni" opportuna data è equivalente alla seguente:
$$ f: A \to B \ continua \ in \ x \in A \ se 

\forall \varepsilon > 0 \exists \delta \ t.c\ |f(x)-f( \overline x ) | < \varepsilon

\forall x \in A \cap B (x, \delta)
$$
\end{document}
